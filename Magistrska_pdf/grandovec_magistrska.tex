\documentclass[12pt,a4paper]{article}
\usepackage[left=2.2cm,right=2.2cm,bottom=2.8cm,top=2.7cm]{geometry}

\usepackage{mathtools}
\usepackage{amsmath}
\usepackage{wrapfig}
\usepackage{blindtext}
\usepackage{breqn}
\usepackage{siunitx}
\usepackage{amssymb}
\usepackage{color}
\usepackage{float}
\usepackage[english, slovene]{babel}  %language
\usepackage[utf8]{inputenc}
\usepackage[T1]{fontenc}
\usepackage{listings}
\usepackage[titletoc,title]{appendix}
\usepackage{epstopdf}
\usepackage{eurosym}
\usepackage{makeidx}
\usepackage{cancel}
\usepackage{floatflt}
\usepackage{sidecap}
\usepackage{amsthm}
\usepackage{enumitem}
\usepackage[bottom]{footmisc}
\usepackage{bbold}
\usepackage{datetime}
\usepackage{floatflt}
\usepackage[dvipsnames]{xcolor}
\usepackage[notbib,nottoc]{tocbibind}
\usepackage{lmodern}
\usepackage{setspace}

\usepackage{titletoc}
\usepackage{calc}
\newlength{\tocindent}

\settowidth{\tocindent}{2.12.2}% calculation of the indentation
\titlecontents{section}[\dimexpr \tocindent-1.1em]{}
{\bfseries\contentslabel[{\thecontentslabel}]{\dimexpr \tocindent-1.0em}}{}
{\titlerule*[8pt]{.}\contentspage}
\titlecontents{subsection}[\dimexpr \tocindent+1.45em]{}
{\contentslabel[{\thecontentslabel}]{\dimexpr \tocindent}}{}
{\titlerule*[8pt]{.}\contentspage}
\titlecontents{subsubsection}[\dimexpr \tocindent+4.95em]{}
{\contentslabel[{\thecontentslabel}]{\dimexpr \tocindent+1.0em}}{}
{\titlerule*[8pt]{.}\contentspage}

\usepackage{titlesec}
\titleformat*{\section}{\normalsize\bfseries}
\titleformat*{\subsection}{\normalsize\bfseries}
\titleformat*{\subsubsection}{\normalsize\bfseries}

\usepackage{hyperref}  %import last
\hypersetup{
    colorlinks=true,
    allcolors=blue,
    % linkcolor=black,
    linktocpage=True,
    allbordercolors={1 1 1}
    }

\newdateformat{monthyear}{\monthname[\THEMONTH] \THEYEAR}
\DeclareGraphicsExtensions{.pdf,.png}
\setcounter{topnumber}{4}
\setcounter{bottomnumber}{4}
\setcounter{totalnumber}{5}
\setlength\tabcolsep{10pt}
\setlength\parindent{12pt}
\setlength\parskip{3pt}
\setlength\jot{0.8em}

\newcommand{\intt}{\int\displaylimits}
\newcommand{\summ}{\sum\displaylimits}
\newcommand{\prodd}{\prod\displaylimits}
\newcommand{\half}{\frac{1}{2}}
\newcommand{\dd}{\mathrm{d}}
\newcommand{\pd}{\partial}
\newcommand{\DD}{\mathcal{D}}
\newcommand{\Dd}[2]{\frac{\dd #1}{\dd #2}}
\newcommand{\Pd}[2]{\frac{\pd #1}{\pd #2}}
\newcommand{\Fd}[2]{\frac{\delta #1}{\delta #2}}
\newcommand{\ee}[1]{\cdot 10^{#1}}
\newcommand{\Lagr}{\mathcal{L}}
\newcommand{\Hamilt}{\mathcal{H}}
\newcommand*\dAlambert{\mathop{}\!\mathbin\Box}
\newcommand{\munu}{{\mu \nu}}
\newcommand{\qu}[1]{``#1''}
\newcommand{\tr}{\mathrm{tr}}
\newcommand{\Tr}{\mathrm{Tr}}
\renewcommand{\div}{\mathrm{div}}
\newcommand{\vct}[1]{\textbf{#1}}
\newcommand{\ket}[1]{|#1 \rangle}
\newcommand{\bra}[1]{\langle #1 |}
\newcommand{\expval}[1]{\langle #1 \rangle}
\newcommand{\Expval}[2]{\langle #1 |#2 \rangle}

% \def\varphi{\phi}
\def\epsilon{\varepsilon}
% \def\theta{\vartheta}
\numberwithin{equation}{section}
%-----------------------------------------------------------------------------
%-----------------------------------------------------------------------------
\begin{document}
\begin{titlepage}

    \vspace*{-2.7cm}
    \begin{center}
	{\large UNIVERZA V LJUBLJANI} \par
	{\large FAKULTETA ZA MATEMATIKO IN FIZIKO} \par
	{\large ODDELEK ZA FIZIKO} \par
	\vspace{4cm}
	{\large Uroš \textsc{Grandovec}} \par
	\vspace{1cm}
	% \includegraphics[width=0.52\columnwidth]{fmf.pdf} \par
    {\setstretch{1.1} % Increase line spacing to 1.5
    \Large\bfseries PREDNOSTI UPORABE REKURENTNIH NEVRONSKIH MREŽ PRI SIMULACIJI FERMI-PASTA-ULAM-TSINGOU SISTEMA
    \par}
    % \rule{15cm}{0.05cm} \par
    \vspace{1cm}
    {\large Magistrsko delo} \par
    \vspace{1cm}
	{\large Mentor: dr. Martin \textsc{Horvat}} \par
	\vfill

	{\large Ljubljana, \monthyear\today}
    \end{center}

\end{titlepage}

%-----------------------------------------------------------------------------
\newpage

\hspace{-0.75cm}
\rule{\columnwidth}{0.02cm}
\vspace{-0.85cm}
\tableofcontents
\hspace{-0.75cm}
\rule{\columnwidth}{0.02cm}

% \pagestyle{headings}

\newtheorem{theorem}{Theorem}

\theoremstyle{definition}
\newtheorem*{definition}{Definition}
\newtheorem*{example}{Example}
\newtheorem*{remark}{Note}

% \theoremstyle{remark}

\section{Uvod}

uporaba:
\section{FPUT sistem}

Obravnavali bomo sistem Fermi-Pasta-Ulam-Tsingou (FPUT). Poznamo tudi obliki $\alpha$-FPUT ($\alpha \neq 0$, $\beta = 0$) in $\beta$-FPUT ($\alpha = 0$, $\beta \neq 0$) sistema.

\begin{equation}
	H = \sum_{i=0}^{N} \frac{p_i^2}{2}+ \sum_{i=0}^{N} \left[ \half (q_{i+1}-q_i)^2 + \frac{\alpha}{3} (q_{i+1}-q_i)^3 + \frac{\beta}{4} (q_{i+1}-q_i)^4 \right]
\end{equation}

Hamiltonove enačbe so potem

\begin{equation}
\begin{split}
	\dot{q}_i &= \Pd{H}{p_i}=p_i,\\ 
    \dot{p}_i &= -\Pd{H}{q_i} \\
    &=(q_{j+1}-q_j)(\delta^i_j-\delta^i_{j+1})+ \alpha \, (q_{j+1}-q_j)^2 (\delta^i_j-\delta^i_{j+1}) \\
    & \quad + \beta \, (q_{j+1}-q_j)^3 (\delta^i_j-\delta^i_{j+1}) \\
    &=(q_{i+1}-q_i) - (q_{i}-q_{i-1}) + \alpha \left[ (q_{i+1}-q_i)^2- (q_{i}-q_{i-1})^2 \right] \\
    &\quad + \beta \left[ (q_{i+1}-q_i)^3- (q_{i}-q_{i-1})^3 \right]
\end{split}
\end{equation}

\section{Rekurentne nevronske mreže}

\begin{figure}[H]
    \centering
    \includegraphics[scale=0.8]{D:/School/Magistrska/images/energyConservationQuote.png}
    \caption{aa.}
\end{figure}\par

%----------------------------------------------------------------

% \bibliographystyle{plain}
% \bibliographystyle{apacite}

% \begin{thebibliography}{5}
%
% \bibitem{Timotej}
% S. Grozdanov, T. Lemut.
% \textit{Reconstruction of spectra and an algorithm based on the theorems of Darboux and Puiseux}
% arXiv preprint arXiv:2209.02788, 2022.


% \end{thebibliography}

\end{document}
